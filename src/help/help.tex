\chapter{HELP!}

One of the nice things about Unix is the on-line help system. By on-line I do
not mean on the Internet (though you can find it there too)
but the documentation of the system installed on your local system. 
It is the one facility that you should familiarze yourself with as you will use it a lot.
The quality and breadth of the on-line documentation varries from distribution. 
 
Back when I first started University there was a table with a whole lot of paper
and books fastened to a table in the computer lab.  I spent a lot of time
thumbing through the pages while I waited for my console to log in.  It was
here I found the basic on-line documenation.   These were referred to as 
{\em man pages} which is short for {\em manual pages}.  In reality those
printed pages were the literal print outs of the on-line documentation readily
available on the system.

There is a deep temptation to "just google it", and you can learn a lot
from the results you'll find there on sites such as stackoverflow or
serverfault. The danger here is that you will see the solution, use the
solution, but not understand the solution. Understanding the solution and
being able to do it again is far more valuable than being able to do a task
once.

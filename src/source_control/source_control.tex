\chapter[Source Control]{}
% http://basepath.com/aup/talks/SCCS-Slideshow.pdf

Source control was invented as a means of mitigating a programmer's innate ability to shoot themselves in the foot.
To say software changes is an understatement of the highest order, almost to the point of being ridiculous.  I would
guess that every single person who has written software has at one point made changes to their code and regretted it 
shortly after.  Everyone does this and it's normal.  Like so many things knowing about a danger is important but not 
nearly as important as what you are going to do about it.

\subsection{Poor Person's Source Control}{}

In the absence of other tools (and before other tools were even around) the practice was to make copies of everything
at frequent (and sometimesly annoyingly frequent) intervals.  We refer to this as the``Poor Person's Source Control."
At every step in the progression of your work you make a simple backup of what you are working on before proceeding and
put it somewhere safe.

Setting it up is as simple a little script:
	{\tt tar -zcf ~/backups/myProgram-`date +\%Y\%m\%d-\%H\%M\%s` ~/myProgram/}

The above `one-liner' is pretty simple when decomposed.  The stuff enclosed in the `back-ticks' ({\tt `}) is executing the date command which outputs the date and time in a specific format.  That result is passed as part of the arguments to the 'tar' command which combines and compresses the contents of the folder named {\tt myProgram}.  The simple approach from here is to put this
into put this into a little script which you run before each functional iteration (ie. before you start work) or if you are particularly paranoid you can put this in your login script to backup every time you start a login shell.  At the very least
what you are doing is making a full backup of your work before you start working which you can use to recover from a bad editing
session or to refer back to a previous iteration.
